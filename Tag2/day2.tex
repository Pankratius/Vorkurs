\documentclass[11pt]{article}
\usepackage{../vorkurs}
\NewEnviron{killcontents}{}
\let\proof\killcontents
\let\endproof\endkillcontents
\begin{document}
\title{\line(1,0){250}\\Tag 2\\\line(1,0){250}}
\date{}
\author{\itshape Vorkurs Mathematik für Nebenfächler 2018}
\maketitle
\begin{framed}
	\noindent \scriptsize
	\textsc{Symbole und Definitionen:} für $a_1,a_2,...,a_n$ schreiben wir für deren \emph{Summe} $a_1+...+a_n=:\sum_{i=1}^n a_i$ und für deren \emph{Produkt} $a_1\cdot...\cdot a_n =: \prod_{i=1}^n$. Für eine Funktion $f:D\to W,~x\mapsto f(x)$ heißt $D$ \emph{Definitionsbereich} und $W$ \emph{Wertebereich}.
\end{framed}
\section*{Aufgaben}
\begin{task}
An einer Klausur, bei der $40$ Punkte zu erreichen waren und bei der nur ganzzahlige Punkte
vergeben wurden, nahmen Studenten aus $6$ verschiedenen Studieng\"angen teil. Zum Bestehen
waren $16$ Punkte erforderlich. Es bezeichne $a_{ij}$ die Anzahl der Studenten des Studienganges
$i$ ($i = 1, 2, . . ., 6$), die $j$ Punkte erreichten. Dr\"ucken Sie folgende Sachverhalte mithilfe des
Summenzeichens aus:
\begin{enumerate}
	\item An der Klausur nahmen insgesamt $411$ Studenten teil.
	\item $222$ Teilnehmer haben die Klausur nicht bestanden.
	\item $3$ Klausurteilnehmer schafften keinen einzigen Punkt.
	\item $43,1 \%$ der Teilnehmer aus dem Studiengang $6$ haben die Klausur nicht bestanden.
\end{enumerate}
\end{task}
\begin{task}
	Radioaktiver Kohlenstoff $^{14}$C, den man zur Datierung fossiler Funde
	benutzt, hat eine Halbwertszeit vn etwa $5,776$ Jahren. Berechnen Sie, wie viel
	von einem Gramm $^{14}$C nach $10 000$ Jahren noch vorhanden ist. Nach wie vielen Jahren ist noch ein Viertel des ursprünglichen Materials vorhanden?
\end{task}
\begin{task}
Sei $f(x)=x^2+24x+128$ und $g(x)=3x+2$. Ermitteln Sie die Funktionsvorschriften $(f \circ g)(x)$ und $(g \circ f)(x)$ sowie die Definitions- und Bildbereiche von $f, g, f \circ g, g \circ f$ und skizzieren Sie $g$.
\end{task}
\begin{task}
	Warum muss das Polynom $p:\mathbb{R}\to \mathbb{R},x\mapsto -28x^2+177$ zwei Nullstellen haben? Skizzieren Sie die Funktion.
\end{task}
\begin{task}
\begin{enumerate}
	\item $f(x)=(x+8)^2+(x-8)^2$
	\item $f(x)=(x+8)^3+(x-8)^3$
	\item $f(x)=\sin(x+8)+\sin(x-8)$
	\item $f(x)=\sin(x+8)-\sin(x-8)$
\end{enumerate}
\end{task}
\begin{htask}
	Zeigen Sie, dass die Funktion $f$ mit $f(x)=1+\frac{\sqrt{x}}{\sqrt{x+1}}$, $x \geq 0$ eine Umkehrfunktion besitzt. Ermitteln Sie diese Umkehrfunktion und ihren Definitions- und Bildbereich.
\end{htask}

\hrule
\vspace{.5cm}
\noindent
\textsc{Hinweise:}\\
Schwierigere Aufgaben, bei denen man vielleicht auch nicht direkt einen Bezug zur Vorlesung erkennt, sind mit einem $\dagger$ gekennzeichnet.\\
Wir versuchen, die Aufgaben und einige Lösungen unter \url{https://pankratius.github.io} zur Verfügungn zu stellen.
Viele Aufgaben sind folgender Literatur entnommen:
\begin{itemize}
	\item ``Br\"uckenkurs Mathematik f\"ur Studieneinsteiger aller Disziplinen'', G. Walz, F. Zeilfelder, Th. Rie\ss inger, Spektrum Verlag, 1. Auflage, 2005
	\item ``Aufgabensammlung zur H\"oheren Mathematik mit ausf\"uhrlichen L\"osungen'' von Dr. Rolf Haftmann, TU Chemnitz.
\end{itemize}
\end{document}
