\documentclass[11pt]{article}
\usepackage{../vorkurs}
\NewEnviron{killcontents}{}
\let\proof\killcontents
\let\endproof\endkillcontents
\begin{document}
\title{\line(1,0){250}\\Tag 3\\\line(1,0){250}}
\date{}
\author{\itshape Vorkurs Mathematik für Nebenfächler 2018}
\maketitle
\section*{Aufgaben}
\textsc{Wichtig:} Wir empfehlen, bei allen Aufgaben, soweit möglich, auf die Benutzung von CAS-Rechnern zu verzichten!
\begin{task}
	In einem rechtwinkligen Dreieck sei die Summe der L\"ange der Katheten
	$2$ und die Hypothenuse sei doppelt so lang wie eine der beiden Katheten.
	Bestimmen Sie die L\"angen der Katheten und die L\"ange der Hypothenuse des
	Dreiecks.
\end{task}
\begin{task}
	Gegeben sei ein Dreieck mit $a=2$, $b=8$ und dem Winkel $\alpha=10^\circ$,
	der $a$ gegen\"uberliegt. Bestimmen Sie die verbleibende Seitenl\"ange $c$ und
	die verbleibenden Innenwinkel $\beta$ und $\gamma$. Betrachten Sie jetzt
	dieselben vorgegebenen Seitenl\"angen, aber $\alpha=40^\circ$. Was beobachten
	Sie?
\end{task}
\begin{task}
	 In einem Parallelogramm seien die Seitenl\"angen $a=5$ und $b=3$ sowie
	 der Winkel $\beta=120^\circ$ ($\beta$ sei der Winkel zwischen den
	 Seiten der L\"ange $b$ und $a$ gegen den Uhrzeigersinn) vorgegeben. Berechnen
	 Sie die L\"ange der Diagonalen und den Fl\"acheninhalt des Parallelogramms.
\end{task}
\begin{task}
	Freitag ist wie immer gro\ss er Pizzatag und Sie bestellen f\"ur sich
	und Ihre sieben Freunde eine Pizza mit $80$ cm Durchmesser. Wie viel cm$^2$
	Pizza bekommt jeder bei gerechter Aufteilung? Wieviel cm unbelegter Rand steht
	jedem zur Verf\"ugung? Der Pizzab\"acker macht Ihnen ein Angebot: Er verkauft
	Ihnen zum selben Preis eine Pizza in Form eines Kreisrings mit \"au\ss erem
	Durchmesser $1$ Meter und innerem Durchmesser von $50$ cm. Berechnen Sie wieder
	die Fl\"ache und die Randl\"ange, die jeder erh\"alt. Nehmen Sie das Angebot an?
\end{task}
\begin{task}
Ein $4$ cm hoher Gegenstand befindet sich $20$ cm vor eine konvexen Linse mit einer Brennweite von $f=+12$ cm. Bestimmen Sie Lage und H\"ohe des durch die Linse erzeugten Bildes. 
Erstellen Sie dazu eine geeignete Skizze. Was \"andert sich, wenn sich der Gegenstand nur $10$ cm von der Linse entfernt befindet?
\end{task}
\begin{htask}
	Ermitteln Sie durch Betrachtung der Winkel im gleichseitigen bzw. im gleichsschenkligen rechtwinkligen Dreieck den Sinus, Kosinus und Tangens von $30^\circ$, $45^\circ$ und $60^\circ$!
\end{htask}
\hrule
\vspace{.5cm}
\noindent
\textsc{Hinweise:}\\
Schwierigere Aufgaben, bei denen man vielleicht auch nicht direkt einen Bezug zur Vorlesung erkennt, sind mit einem $\dagger$ gekennzeichnet.\\
Wir versuchen, die Aufgaben und einige Lösungen unter \url{https://pankratius.github.io} zur Verfügung zu stellen.
Viele Aufgaben sind folgender Literatur entnommen:
\begin{itemize}
	\item ``Br\"uckenkurs Mathematik f\"ur Studieneinsteiger aller Disziplinen'', G. Walz, F. Zeilfelder, Th. Rie\ss inger, Spektrum Verlag, 1. Auflage, 2005
	\item ``Aufgabensammlung zur H\"oheren Mathematik mit ausf\"uhrlichen L\"osungen'' von Dr. Rolf Haftmann, TU Chemnitz.
\end{itemize}
\end{document}
