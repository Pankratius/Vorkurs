\documentclass[11pt]{article}
\usepackage{../vorkurs}
\NewEnviron{killcontents}{}
\let\proof\killcontents
\let\endproof\endkillcontents
\begin{document}
\title{\line(1,0){250}\\Tag 4\\\line(1,0){250}}
\date{}
\author{\itshape Vorkurs Mathematik für Nebenfächler 2018}
\maketitle
\section*{Aufgaben}
\textsc{Wichtig:} Wir empfehlen, bei allen Aufgaben, soweit möglich, auf die Benutzung von CAS-Rechnern zu verzichten!
\begin{task}
	Lösen Sie die Gleichungen:
	\begin{enumerate}[i)]
		\item $\frac{2x-1}{2-x} = \frac{7}{3x+4}$
		\item $\frac{x+1}{2x-4} = \frac{x+2}{x-2}$
	\end{enumerate}
\end{task}
\begin{task}
	Finden Sie alle Nullstellen von $x^4-x^3-31x^2+x+30=0$!
\end{task}
\dotfill
\begin{task}
	Um die Funktion der Bauchspeicheldr\"use zu testen, wird ein bestimmter Farbstoff in sie eingespritzt und dessen Ausscheiden gemessen. Eine gesunde Bauchspeicheldr\"use
	scheidet pro Minute $4 \%$ des jeweils noch vorhandenen Farbstoffs aus. Bei einer Untersuchung wird einem Patienten $0,2$ Gramm des Farbstoffes injiziert. Nach
	$30$ Minuten sind noch $0,09$ Gramm des Farbstoffes in seiner Bauchspeicheldr\"use vorhanden. Funktioniert seine Bauchspeicheldr\"use normal?
\end{task}
\begin{task}
	Eistee kann einen Koffeingehalt von $50$ Milligramm pro $0,33$l Dose haben. Bei einem Jugendlichen setzt die Wirkung des Koffeins nach ca. 1 Stunde ein. Der Koffeingehalt im Blut nimmt dann exponentiell
	mit einer Halbwertszeit von $3$ Stunden ab. Eine B\"uchse Eistee enth\"alt $50$mg Koffein. Wann sind nur noch $0,01$mg Koffein im Blut vorhanden, wenn der Abbau ca. 1 Stunde nach dem Verzehr beginnt?
\end{task}
\dotfill
\begin{task}
Bestimmen Sie die L\"osungen der folgenden Wurzelgleichungen: 
\begin{enumerate}[i)]
	\item $\sqrt{x+4}=\sqrt{x^2+x}$
	\item $\sqrt{x}+\sqrt{5-x}=\sqrt{2x+7}$
\end{enumerate}  
\end{task}
\hrule
\vspace{.5cm}
\noindent
\textsc{Hinweise:}\\
Schwierigere Aufgaben, bei denen man vielleicht auch nicht direkt einen Bezug zur Vorlesung erkennt, sind mit einem $\dagger$ gekennzeichnet.\\
Wir versuchen, die Aufgaben und einige Lösungen unter \url{https://pankratius.github.io} zur Verfügung zu stellen.\\
Viele Aufgaben sind folgender Literatur entnommen:
\begin{itemize}
	\item ``Br\"uckenkurs Mathematik f\"ur Studieneinsteiger aller Disziplinen'', G. Walz, F. Zeilfelder, Th. Rie\ss inger, Spektrum Verlag, 1. Auflage, 2005
	\item ``Aufgabensammlung zur H\"oheren Mathematik mit ausf\"uhrlichen L\"osungen'' von Dr. Rolf Haftmann, TU Chemnitz.
\end{itemize}
\end{document}
