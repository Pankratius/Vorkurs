\documentclass[11pt]{article}
\usepackage{../vorkurs}
\NewEnviron{killcontents}{}
\let\proof\killcontents
\let\endproof\endkillcontents
\begin{document}
\title{\line(1,0){250}\\Tag 5\\\line(1,0){250}}
\date{}
\author{\itshape Vorkurs Mathematik für Nebenfächler 2018}
\maketitle

\section*{Aufgaben}
\textsc{Wichtig:} Wir empfehlen, bei allen Aufgaben, soweit möglich, auf die Benutzung von CAS-Rechnern zu verzichten!
\begin{task}
	Bestimmen Sie die L\"osungsmengen der folgenden Ungleichungen: 
	\begin{enumerate}[i)]
		\item $2x+1<3x-5$
		\item $2(x+1)(x-2)<2x^2+5$
		\item $|2x-4|<8$
		\item $|x-4|+|x+8|<24$
	\end{enumerate}
\end{task}
\begin{task}
	Drei Freundinnen wollen einen Wochenendausflug machen und zu diesem Zweck ein Auto mieten.
	Unternehmen $A$ bietet einen entsprechenden PKW zu einem Gruppenpreis von $80,00$ Euro an zuz\"uglich $0,20$ Euro je gefahrenen $km$, w\"ahrend bei Unternehmen $B$ neben
	$44,00$ Euro Grundgeb\"uhr $0,35$ Euro je $km$ zu zahlen sind. Bei welchen Fahrstrecken ist es g\"unstiger, bei $A$ zu mieten, und bei welchen Strecken ist $B$ vorteilhafter?
\end{task}
\dotfill
\begin{task}
	Selllen Sie die Mengen
	\[
	\mathcal{A}=\{(x,y)~|~x\in \mathbb{R},y\in \mathbb{R},|x|+|y| = 1\},~\mathcal{B} = \{(x,y)~|~x\in \mathbb{R},y\in \mathbb{R},|x|+|y|\leq 1\}
	\]
	graphisch in einem entsprechenden Koordinatensystem dar.
\end{task}
\dotfill
\begin{task}
Aus Unachtsamkeit wird einem Patienten die $2,5$-fache Menge eines Medikamentes gespritzt. Er soll daher so lange unter medizinischer Kontrolle bleiben, bis sich im K\"orper nur noch die urspr\"unglich vorgesehene Dosis von $2$ml befindet. Es wird davon ausgegangen, dass pro Stunde etwa $4\%$ des im K\"orper befindlichen Medikaments abgebaut und ausgeschieden werden. Nach wie vielen Stunden ist im K\"orper des Patienten nur noch die Normaldosis - $2$ml - enthalten?
\end{task}
\begin{task}
	Bestimmen Sie die L\"osungen der folgenden Wurzelgleichung:
	\begin{enumerate}[i)]
		\item $\sqrt{x-\sqrt{x+2}}=2$
	\end{enumerate}  
\end{task}
\hrule
\vspace{.5cm}
\noindent
\textsc{Hinweise:}\\
Schwierigere Aufgaben, bei denen man vielleicht auch nicht direkt einen Bezug zur Vorlesung erkennt, sind mit einem $\dagger$ gekennzeichnet.\\
Wir versuchen, die Aufgaben und einige Lösungen unter \url{https://pankratius.github.io} zur Verfügung zu stellen.\\
Viele Aufgaben sind folgender Literatur entnommen:
\begin{itemize}
	\item ``Br\"uckenkurs Mathematik f\"ur Studieneinsteiger aller Disziplinen'', G. Walz, F. Zeilfelder, Th. Rie\ss inger, Spektrum Verlag, 1. Auflage, 2005
	\item ``Aufgabensammlung zur H\"oheren Mathematik mit ausf\"uhrlichen L\"osungen'' von Dr. Rolf Haftmann, TU Chemnitz.
\end{itemize}
\end{document}
