\documentclass[11pt]{article}
\usepackage{../vorkurs}
\NewEnviron{killcontents}{}
\let\proof\killcontents
\let\endproof\endkillcontents
\begin{document}
\title{\line(1,0){250}\\Tag 7\\\line(1,0){250}}
\date{}
\author{\itshape Vorkurs Mathematik für Nebenfächler 2018}
\maketitle

\section*{Aufgaben}
\textsc{Wichtig:} Wir empfehlen, bei allen Aufgaben, soweit möglich, auf die Benutzung von CAS-Rechnern zu verzichten!
\begin{task}
	Zeigen Sie, dass $f(x)=e^{x^2-x}$ genau ein lokales Extremum besitzt und bestimmen Sie den zugeh\" origen Extrempunkt des Graphen von $f$.
\end{task}
\begin{task}
	Seien $a,b\in \mathbb{R}$ und $f:\mathbb{R}\to \mathbb{R},~x\mapsto x^2+ax+b$. Bestimmen Sie in Abhängigkeit der Parameter $a$ und $b$ die Extremstellen von $f$.
\end{task}
\begin{task}
	Bestimmen Sie die Extrema von $f:\mathbb{R}\to \mathbb{R},~x\mapsto x^2\cdot e^{-x^2}$. Hat $f$ lokale Minima?
\end{task}
\dotfill
\begin{task}
	Stellen Sie fest, ob $f(x)=\frac{1}{12}x^4$ eine Wendestelle besitzt.
\end{task}
\begin{task}
	Zeigen Sie, dass $f(x)= e^{x^2-x}$ keine Wendestelle hat.
\end{task}

\hrule
\vspace{.5cm}
\noindent
\textsc{Hinweise:}\\
Schwierigere Aufgaben, bei denen man vielleicht auch nicht direkt einen Bezug zur Vorlesung erkennt, sind mit einem $\dagger$ gekennzeichnet.\\
Wir versuchen, die Aufgaben und einige Lösungen unter \url{https://pankratius.github.io} zur Verfügung zu stellen.\\
Viele Aufgaben sind folgender Literatur entnommen:
\begin{itemize}
	\item ``Br\"uckenkurs Mathematik f\"ur Studieneinsteiger aller Disziplinen'', G. Walz, F. Zeilfelder, Th. Rie\ss inger, Spektrum Verlag, 1. Auflage, 2005
	\item ``Aufgabensammlung zur H\"oheren Mathematik mit ausf\"uhrlichen L\"osungen'' von Dr. Rolf Haftmann, TU Chemnitz.
\end{itemize}
\end{document}
