\documentclass[11pt]{article}
\usepackage{../vorkurs}
\NewEnviron{killcontents}{}
\let\proof\killcontents
\let\endproof\endkillcontents
\begin{document}
\title{\line(1,0){250}\\Tag 6\\\line(1,0){250}}
\date{}
\author{\itshape Vorkurs Mathematik für Nebenfächler 2018}
\maketitle

\section*{Aufgaben}
\textsc{Wichtig:} Wir empfehlen, bei allen Aufgaben, soweit möglich, auf die Benutzung von CAS-Rechnern zu verzichten!
\begin{task}
	Stellen Sie den Vektor $\begin{pmatrix}1\\4\\5\end{pmatrix}$ als Linearkombination der Vektoren 
	$\begin{pmatrix}1\\2\\-1\end{pmatrix}$, $\begin{pmatrix}0\\1\\5\end{pmatrix}$ und $\begin{pmatrix}0\\0\\2\end{pmatrix}$ dar.
\end{task}
\begin{task}
Bestimmen Sie, ob die folgenden Vektoren linear unabh\" angig sind:
		\[\begin{pmatrix}1\\1\\-1 \end{pmatrix},
		\begin{pmatrix}2\\ 3\\-1 \end{pmatrix}~\text{und}~
		\begin{pmatrix}-1\\ 0\\2 \end{pmatrix}.\]
\end{task}
\begin{task}
	Charakterisieren Sie die Lösungsmengen der folgenden Gleichungssysteme.\\
	\begin{minipage}{0.3\textwidth}
		$$ x+2y+z=1$$
		$$ 5y+6z=2$$
		$$9z = 3$$
	\end{minipage}
	\begin{minipage}{0.3\textwidth}
		$$ x+2y+3z=1$$
		$$ 5y+6z=2$$
	\end{minipage}
	\begin{minipage}{0.3\textwidth}
	 $$x+2y+3z=1$$
	 $$5y+6z=2$$
	 $$0=3$$
	\end{minipage}
\end{task}
\begin{task}
	Gesucht sind zwei reelle Zahlen mit folgenden Eigenschaften: Ihre Summe ist $4$. Vermindert man das Dreifache der einen Zahl um das Doppelte der anderen Zahl, so erh\"alt man $52$.
\end{task}
\begin{task}
	F\"ur die Vorbereitung von insgesamt $30$ Fr\"uhst\"ucksgedecken sollen $54$ Portionspackungen Wurst, $88$ Portionspackungen K\"ase und $62$ Portionspackungen Marmelade verwendet werden. 
	F\"ur die einzelnen Gedecke werden ben\"otigt:
	
	Gedeck A: 1 Wurst, 3 K\"ase, 3 Marmelade;
	
	Gedeck B: 1 Wurst, 4 K\"ase, 2 Marmelade;
	
	Gedeck C: 3 Wurst, 2 K\"ase, 1 Marmelade;
	
	Gedeck D: 4 Wurst, 1 K\"ase, 2 Marmelade.
	
	Welche Anzahl der einzelnen Gedecke kann vorbereitet werden? 
\end{task}
\begin{task}
	F\"ur einen Flug werden Tickets in den Bef\"orderungsklassen Economy und Business angeboten.
	Die $300$ Economypl\"atze werden zu unterschiedlichen Sonderkonditionen zu Preisen von $20$ Euro und $220$ Euro sowie zum Normalpreis von $420$ Euro verkauft. Die $50$ Businesspl\"atze werden zu
	Sonderkonditionen zum Preis von $600$ Euro und zum Normalpreis von $1000$ Euro verkauft. Zu den beiden Normalpreisen werden zusammen $100$ Tickets verkauft.
	Geben Sie alle m\"oglichen L\"osungen daf\"ur an, wie viele Tickets der einzelnen Preiskategorien verkauft werden m\"ussen, um bei voll besetztem Flugzeug einen Erl\"os von insgesamt
	$124 \ 000$ Euro zu erzielen! 
\end{task}
\hrule
\vspace{.5cm}
\noindent
\textsc{Hinweise:}\\
Schwierigere Aufgaben, bei denen man vielleicht auch nicht direkt einen Bezug zur Vorlesung erkennt, sind mit einem $\dagger$ gekennzeichnet.\\
Wir versuchen, die Aufgaben und einige Lösungen unter \url{https://pankratius.github.io} zur Verfügung zu stellen.\\
Viele Aufgaben sind folgender Literatur entnommen:
\begin{itemize}
	\item ``Br\"uckenkurs Mathematik f\"ur Studieneinsteiger aller Disziplinen'', G. Walz, F. Zeilfelder, Th. Rie\ss inger, Spektrum Verlag, 1. Auflage, 2005
	\item ``Aufgabensammlung zur H\"oheren Mathematik mit ausf\"uhrlichen L\"osungen'' von Dr. Rolf Haftmann, TU Chemnitz.
\end{itemize}
\end{document}
