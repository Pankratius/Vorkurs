\documentclass[11pt]{article}
\usepackage{amsmath,amssymb,amsthm}
\usepackage{graphicx}
\usepackage[left=2cm,right=2cm,top=2cm,bottom=2cm]{geometry}
\usepackage[german]{babel}
\usepackage[utf8x]{inputenc}
\newtheoremstyle{generalpurposedef}
{1em}
{1em}
{}
{}
{\bfseries}
{:}
{.5em}
{}
%------
%use for theorems and such
\theoremstyle{generalpurposedef}
\newtheorem{task}{Aufgabe}
\newtheoremstyle{generalpurposetsk}
{1em}
{1em}
{}
{}
{\bfseries}
{$\dagger$:}
{.5em}
{}
%------
%use for theorems and such
\theoremstyle{generalpurposetsk}
\newtheorem{htask}[task]{Aufgabe}
\addto\captionsgerman{\renewcommand\proofname{Lösung}}
\renewcommand\qedsymbol{}
\usepackage{environ}
\NewEnviron{killcontents}{}
\let\proof\killcontents
\let\endproof\endkillcontents
\usepackage{framed}
\usepackage{dashrule}


\begin{document}
\title{\line(1,0){250}\\Tag 1\\\line(1,0){250}}
\date{}
\author{\itshape Vorkurs Mathematik für Nebenfächler 2018}
\maketitle
\begin{framed}
	\noindent \scriptsize
	\textsc{Symbole und Definitionen:} $\mathbb{N} = \{1,2,3,...\}$(natürliche Zahlen), $\mathbb{Z}:= \{...,-2,-1,0,1,2,...\}$ (ganze Zahlen), $\mathbb{Q} := \{p/q~|~p,q\in \mathbb{Z},q\neq 0\}$ (rationale Zahlen)
\end{framed}
\section*{Aufgaben}
\begin{task}
	$8$ Maschinen erledigen eine Arbeit in $5$ Tagen. Wie lange brauchen $10$ Maschinen f\"ur dieselbe Aufgabe?
\end{task}
\begin{task}
	Ein $40$cm langer Draht vom Durchmesser $4$mm hat die Masse $36,7$g. Wieviel Meter Draht vom gleichen Material, aber vom Durchmesser $6$mm haben die Masse $90$kg? (Haftmann, 1.3)
\end{task}
\begin{task}
	Draht aus gleichem Material, aber von unterschiedlichem Durchmesser, wird mit den Angaben $120$g/m und $85$g/m angeboten. Die zuerst genannte Sorte hat einen Durchmesser von $5$mm. Welchen Durchmesser hat die zweite Sorte?
\end{task}
\begin{task}
	$15$ Kugeln mit einem Umfang von $70$cm wiegen $6,5$kg. Wieviel wiegen $25$ Kugeln aus gleichem Material mit einem Umfang von $60$cm? (Haftmann 1.5)
\end{task}
\begin{task}
	Die Schallgeschwindigkeit in Luft betr\"agt $345 \frac{m}{s}$ bei $24^\circ$C. Der Einfluss der Temperatur auf die Schallgeschwindigkeit in Gasen wird durch die Gleichung
	\[
	\frac{v_1}{v_2}=\sqrt{\frac{T_1}{T_2}}
	\]
	bestimmt, wobei $v_1$ und $v_2$ die Geschwindigkeiten bei den absoluten Temperaturen $T_1$ und $T_2$ sind. Bestimmen Sie die Schallgeschwindigkeit in Luft bei einer Temperatur von $30^\circ$C. (Hinweis: Absolute Temperaturen werden in Kelvin angegeben. Dabei gilt $T_K-273,15=T_C$.)
\end{task}
\dotfill
\begin{task}
Sch\"uler haben eine quadratische Fl\"ache bemalt. Da es sehr sch\"on geworden ist, d\"urfen sie die vier Quadratseiten um $8m$ verl\"angern. Dies bedeutet eine Vergr\"o\ss erung der Fl\"ache um $336m^2$. Welche Ma\ss e haben die alte und die neue bemalte Fl\"ache?
\end{task}
\begin{htask}
	Tim behauptet: ``Addierst Du eine positive rationale Zahl (ungleich Null) zu deren Kehrwert, so hat das Ergebnis mindestens den Wert $2$.'' Beweisen Sie diese Behauptung.
\end{htask}
\begin{proof}
	Wir bezeichnen, wie in der Vorlesung, die Menge der rationalen Zahlen mit $\mathbb{Q}$. Dann gibt es für jede Zahl $p\in \mathbb{Q}$ ganze Zahlen $a,b\in \mathbb{Z}$, sodass $p = a/b$. Für den Kehrwehrt gilt dann $1/p = b/a$. Weil $p>0$ gelten soll, können wir annehmen, dass $a,b>0$ sind. Für die Summe von $a,b$ gilt dann
	\[
	1+\frac{1}{p} =  \frac{a}{b} + \frac{b}{a}= \frac{a^2+b^2}{ab} = \frac{a^2+b^2+2ab-2ab}{ab} \overset{\ast}{=} \frac{(a+b)^2+2ab}{ab} = \underbrace{\frac{(a+b)^2}{ab}}_{\geq 0} + 2 \geq 2,
	\]
	wobei in $\ast$ die erste binomische Formel genutzt wurde.
	
\end{proof}	
\hrule
\vspace{.5cm}
\noindent
\textsc{Hinweise:}\\
Schwierigere Aufgaben, bei denen man vielleicht auch nicht direkt einen Bezug zur Vorlesung erkennt, sind mit einem $\dagger$ gekennzeichnet.\\
Viele Aufgaben sind folgender Literatur entnommen:
\begin{itemize}
	\item ``Br\"uckenkurs Mathematik f\"ur Studieneinsteiger aller Disziplinen'', G. Walz, F. Zeilfelder, Th. Rie\ss inger, Spektrum Verlag, 1. Auflage, 2005
	\item ``Aufgabensammlung zur H\"oheren Mathematik mit ausf\"uhrlichen L\"osungen'' von Dr. Rolf Haftmann, TU Chemnitz.
\end{itemize}
\end{document}
